\section{Extended Model}
\label{sec:extended_model}

The second approach \citeauthor{JERMANNfinancial} employ to analyse the macroeconomic effects of financial shocks is based on the model estimated by \citet{SMETSshocks}. More specifically, they extend this model by including financial shocks and financial frictions as introduced in section \ref{sec:baseline_model}. This allows them to  asses the impact of financial shocks relative to other sources of shocks. 

In contrast to the baseline model, the extended model includes a public sector. Public policy in this model comprises both fiscal and monetary policy. While government purchases and  interest deduction are financed with lump-sum taxes, monetary policy follows some  interest rate rule. The second major deviation from the baseline model are nominal rigidities in wages and prices. Building on the New Keynesian literature, \citeauthor{JERMANNfinancial} incorporate an explicit wage and price setting mechanism of households and firms. By making use of Calvo's price rigidity, they obtain wage rigidities. To generate nominal price rigidities, they employ Rotemberg's approach, which is based on convex price adjustment costs.


\subsection{Estimation of the Extended Model}
\label{sec:extended_model_estimation}

\begin{table}
  \centering
  \resizebox{\textwidth}{!}{
    \caption{\textsc{Parameterization}}
    \begin{tabular}{llccc}
      \toprule \toprule
      Calibrated parameters & & & & Value\\
      \midrule
      \input{../../../bld/out/tables/cal_params.tex} \\[0.5ex]
      \midrule
      Estimated parameters & Prior [mean, std] & Mode & Below 5\% & Below 95 \% \\
      \midrule
      \input{../../../bld/out/tables/est_params.tex}
      \bottomrule    
      \label{tab:table_3}
    \end{tabular}
  }
  \raggedright
  {\footnotesize \textit{The prior distributions for the average price
  markup $\bar{\eta}$ and the average wage markup $\bar{\upsilon}$ are generalized
  beta distributions with support [1, 2] }}
\end{table}


The estimation of the extended model is based on the original data set provided by \citeauthor{JERMANNfinancial}. It contains eight empirical series, namely: GDP, personal consumption expenditures, inflation, federal funds rate, debt repurchases of nonfinancial businesses, working hours and hourly wages. However, to match the empirical series to their theoretical counterparts, the observations need to be transformed prior to the estimation. In particular, estimating the model requires growth rates of the following series: GDP, consumption, investment, price deflator, working hours and wages. Therefore, I first take log-differences of the series above  and demean the resulting series. Given that the federal funds rate is expressed in percentage terms, the original series only needs to be demeaned without taking log-differences. Following the procedure employed in section \ref{sec:baseline_model} for Figure \ref{fig:figure_5}, debt repurchases are linearly detrended.\footnote{Note that debt repurchases as shown in Figure \ref{fig:figure_1} do not seem to have a trend. Hence, linearly detrending the series does not yield significantly different results  compared to demeaning it.}

Table \ref{tab:estimation} shows the parameter values for the calibrated
parameters. For the estimated shocks it shows the prior distribution as well as
the mode and 95\% HPD-interval of the posterior distribution. The posterior
distribution is reached using the Metropolis-Hastings algorithm implemented in dynare.


\subsection{Convergence}
\label{sec:convergence}
