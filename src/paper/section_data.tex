\section{Data}
\label{sec:data}

\begin{figure}[t]
    
    \centering

    \includegraphics[scale=0.7]{../../out/figures/figure_1}

    \caption{\textsc{Financial Flows in the Nonfinancial Business Sector \textit{(Corporate and Noncorporate)}}, 1952:I--2015:III}
    
    \label{fig:figure_1}

\end{figure}

Figure \ref{fig:figure_1} illustrates financial cycles in the US economy. For this purpose, it shows net payments to equity holders and net debt repurchases in the nonfinancial business sector from the first quarter of 1952 to the third quarter of 2015. The data is taken from the Flow of Funds Accounts (FFA) released in December 2015 by the Federal Reserve Board (FRB). References to the FFA are indicated by making use of the ‘Coded Tables’ published on September 18, 2015. The current version of the FFA has undergone some modifications since the original paper was published. Table \ref{table:identifiers} in appendix \ref{sec:appendix_data} contains the relevant changes in identifiers. Additionally, appendix \ref{sec:appendix_data} contains a detailed description of all data sets used for the replication study.

Equity Payout is obtained as described by \citeauthor{JERMANNfinancial}. First, the sum of ‘Net dividends of nonfarm, nonfinancial business’ (FA106121075.Q) and ‘Net dividends of farm business’ (FA136121073.Q) is calculated. This is followed by subtracting ‘Net increase in corporate equities of nonfinancial business’ (FA103164103.Q) and ‘Proprietors’ net investment of nonfinancial business’ (FA112090205.Q). Debt Repurchase is the negative of ‘Net increase in credit markets instruments of nonfinancial business’ (FA144104005.Q). Both variables are divided by Business GDP times 1000. Business GDP is based on data taken from the National Income and Product Accounts (Table 1.3.5) on Business Value Added for the period 1952:I--2015:III. In addition, recessions are highlighted in the figure by the shaded blue areas.

The updated figure generally suggests the same negative correlation between equity payout and debt repurchases as described by \citeauthor{JERMANNfinancial}. In particular, the strong overall relationship between the two variables can be replicated for the period after 1980. This relation also holds true for the newly included values starting from the third quarter of 2010. 
Consequently, the figure supports the interpretation by \citeauthor{JERMANNfinancial}. First, equity and debt appear to be imperfect substitutes. Second, booms are associated with an increase in equity payout while debt repurchases increase during recessions.
